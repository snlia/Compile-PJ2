\documentclass{article}

\usepackage{tmparticle}

\title{《编译原理》课程PJ}
\author{\ldots}
\date{\today}

\begin{document}
\section{项目要求} % (fold)
\label{sec:项目要求}
\begin{enumerate}
  \item 使用flex \& bison 完成对PCAT语言的语法树的建立,并将语法树打印出来
  \item 具有语法报错功能并提示错误位置
\end{enumerate}

% section 项目要求 (end)
\section{实现原理} % (fold)
\label{sec:实现原理}
这部分的实现主要包含两个部分,词法解析和文法解析。

\subsection{词法解析} % (fold)
\label{sub:词法解析}
词法解析主要由flex 实现,代码见tokenizer.h、tokenizer.l。\par
flex 是一个自动化工具,可以按照定义好的规则自动生成一个C 函数yylex(),也称为扫描器。这个C 函数把文本串作为输入,a按照定义好的规则分析文本串中的字符,找到符合规则的一些字符序列后,就执行在规则中定义好的动作。我们定义了行位置变量ln和列位置变量col,以便记录位置。\par
对于关键词(AND|ELSIF|LOOP|PROGRAM|VAR|ARRAY|END|MOD|READ|WHILE|BEGIN|EXIT|NOT|RECORD|WRITE|BY|FOR|OF|RETURN|DIV|IF|OR|THEN|DO|IN|OUT|TO|ELSE|IS|PROCEDURE|TYPE),相应的操作是输出当前行列位置,判断为关键词,增加记录的行号/列号,并保存特定的token信息来构建语法树。\par
对于关键词(AND|ELSIF|LOOP|PROGRAM|VAR|ARRAY|END|MOD|READ|WHILE|BEGIN|EXIT|NOT|RECORD|WRITE|BY|FOR|OF|RETURN|DIV|IF|OR|THEN|DO|IN|OUT|TO|ELSE|IS|PROCEDURE|TYPE),相应的操作是输出当前行列位置,判断为关键词,增加记录的行号/列号,并保存特定的token信息来构建语法树。\par

% subsection 词法解析 (end)
\subsection{文法分析} % (fold)
\label{sub:文法分析}
文法分析主要由bison实现,代码详见main.y。\par
bison是一个语法分析器生成器,可以把一个上下文无关文法的描述通过LALR(1)转化成可以分析该文法的 C 或 C++ 程序。在进行语法分析时,主要是利用其生成的yyparse函数,对FILE*类型的ffin进行语法解析,当发现解析错误时,会调用yyerror函数。这部分主要有三个步骤,文法构建、语法树分析、语法查错。\par
\subsubsection{文法构建} % (fold)
\label{ssub:文法构建}
PCAT语言的文法构建参见其说明文件(第 \number\numexpr\value{page}+1 页)。在进行构建时,要进行两个转化:\par
形如TOKEN\{, TOKEN\}的通过新建一个TOKEN\_block来表示,其中TOKEN\_block定义为:\par
\[
\begin{array}[h]{lll}
  TOKEN\_block & -> & TOKEN\_block ',' TOKEN\\
  & -> & TOKEN
\end{array}
\]
形如[TOKEN]的通过新建一个TOKEN\_opt来表示,其中TOKEN\_opt定义为:\par
\[
\begin{array}[h]{lll}
  TOKEN\_opt & -> & TOKEN\\
  & -> & 
\end{array}
\]
这样,就可以将PCAT的文法转成bison接受的格式,bison会根据这个进行自动分析,在匹配到对应的文法时,执行一定的内容,我们在这里加入语法树的构建即可。

% subsubsection 文法构建 (end)
\subsubsection{语法树构建} % (fold)
\label{ssub:语法树构建}
根据上述的定义,在程序中,每个语法树的节点也就是每个符号,对应了一个类。例如在syntax.h的实现中,class Number对应number类,class UnaryOpExpr对应unary-op类,等等。如果在语法里,符号a能产生符号b,那么在a定义的类就包含一个b对象。当bison产生的语法分析器执行action(写在main.y的语法定义部分)时,语法树就被建立起来。\par


\subsubsection{语法查错} % (fold)
\label{ssub:语法查错}
在syntax.h的最后,对于Program对应的节点,调用void print(int ident) 把整个语法树打印出来。打印的内容详见每一个class中的void print函数代码。\par

% subsubsection 语法树构建 (end)

% subsubsection 语法查错 (end)

% subsection 文法分析 (end)

% section 实现原理 (end)
\section{小组分工} % (fold)
\label{sec:小组分工}
吴韫聪(代码之phrase部分+报告)、朱天歌(代码之语法树部分+报告)、兰石懿(测试)、王禹程(测试)
% section 小组分工 (end)
\end{document}
